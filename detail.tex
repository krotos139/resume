\section{Выполненные задачи}

\jobdetail
{ОАО "СЭЗ им. Серго Орджоникидзе"}
{Начальник бюро разработки ПО}
{Занимался разработкой ПО для авиационной техники.

Решенные задачи:
\begin{itemize}
\item{Разработка BSP для платы пульта ДУ на базе процессора i.MX6. Проект дорогущего пульта для президентского борта. Требовалось подготовить сборку Linux на базе сборочницы Yokto для процессора Freescale i.MX6. В результате выполнения работы потребовалось вносить изменения в u-boot, конфигурацию ядра Linux, DTS. Потребовалось сделать рецепт для сборки lircd для Yokto. Также был разработан драйвер програмного LIRC, микросхемы заряда батареи. Портирован драйвер сенсорной панели. Работы были выполнены командой из 1 человека.}
\item{Разработал драйвер AHCI для VxWorks653. Под моим руководством был разработан драйвер AHCI. В команде был 1 человек. Задача было совершенно новая, пришлось глубоко изучать протокол AHCI, интерфейс PCIe, и работу DMA. Очень долгое время не могли добится ответа от контроллера, не работал DMA канал.}
\item{Разработка комплекта сертификационной документации согласно КТ-178B, DO-178B, DO-178C для ПО изделий МАВИм, ММПм, МГПм. Для выполнения работы в ссжатые сроки отказался от принятой верстки документов в Word, разработал шаблоны документов в Latex, и заполнение данными из электронных таблиц формата CSV и XML документов сгенерированными в Doxygen.}
\end{itemize}
}

%------------------------------------------------

\jobdetail
{ООО "СКТБ "СКиТ"}
{Ведущий инженер программист}
{Занимался разработкой ПО для авиационной техники.

Решенные задачи:
\begin{itemize}
\item{Портирование драйвера МАВИм(Модуль авиационных интерфейсов) на VxWorks653. Портировал драйвер, написал тесты}
\item{Портирование драйвера ММПм(Массовая память) на VxWorks653. Основную работу по портированию выполнил Наконечный Павел. Он нашел реализацию ATA для VxWorks, и сделал обвязку для ее работы в VxWorks653. Я делал интерфейс ввода/вывода через API для приложений.}
\item{Портирование драйвера МГПм(графического процессора) на VxWorks653. При портировании драйвера возникли трудности - API для взаимодействия с устройствами по PCIe не позволяло обращаться по таким адресам внутри окна BAR, так как смещение регистров оказались больше размера выделенного в ОС окна для доступа. По этой причине пришлось гглубоко изучать работу PCIe и дорабатывать функцию изменения смещения окна при записи в регистры.}
\item{Изучение VxWorks653. Доработка BSP Freescale P1010 для поддержки всего функционала ВИМ-3U-3. Разработка драйверов взаимодействия с FPGA. Реализация функции ВСК(встроенных средств контроля). Реализация простой табличной файловой системы и реализация API для взаимодействия приложений с энергонезависимой памятью.}
\item{Проверка мезонина МГПм(графический процессор) под Linux. Для этого графического процессора исходный код драйвера содержал бинарную часть, скомпилированную под x86. По этой причине портирование драйверов под PowerPC не представлялось возможным. Я с помощью gdb захватил порядок записи регистров графического процессора и их аргументы во время прохождения определенного теста, и сделал драйвер позволяющий запустить этот тест под Linux на процессоре PowerPC. Это позволело пройти испытания аппаратуры.}
\item{Проверка мезонина ММПм(массовая память) под Linux. Модернизировал конфиг ядра для включения необходимых функций. ММПм работает через ATA, AHCI.}
\item{Разработка ПО для мезонина МАВИм под Linux. Это плата расширения для ВИМ-3U-3, подключаемая по интерфейсу PCIe. Позволяет принимать и отправлять данные по авиационным интерфейсам - ARINC429, ARINC708, ARINC825. Управление платой происходит через регистры, доступные в адресном пространстве PCIe BAR0. Я разработал драйвер для Linux, который обнаруживает устройство и позволял проверить аппаратуру, предоставлялся интерфейс для пользовательских программ через файлы устройств (ioctl для управления устройствами, read/write для доступа к данным).}
\item{Разработка ПО для стенда проверки ВИМ-3U-3. Стенд был сделан на базе ПК в серверной стойке. Я разрабатывал скрипты для проверки работоспособности аппаратуры в ОС Linux. Для управления тестированием я разаработал графическое приложение на Python+PyQT+PySerial, которое запускает тестовые скрипты и выдает тестовые данные со стенда. Для выдачей тестовых данных использовались готовое оборудование (Ethernet карточки, USB-RS232) и Arduino. ПО для Arduino тоже писал я.}
\item{Портирование и отладка linux для ВИМ-3U-3. Занимался настройкой оборудования используя конфигурацию ядра и DTS файл. Также потребовалось разработать несколько новых драйверов - драйвер взаимодействия с FPGA, EEPROM, NVRAM }
\item Портирование и отладка u-boot для ВИМ-3U-3(Вычислительный модуль на базе Freescale P1010). Портирование заключалось в настройке параметров u-boot и подбор необходимых параметров для работы ОЗУ. Отладку необходимых параметров ОЗУ пришлось вести используя CodeWarrior через интерфейс JTag. Отлаживал аппаратно-программные баги - такие как - не работал Ethernet, при поднятии линка, он вставал на гигабит так как Phy ВИМ-3U-3 и Phy удаленного устройства были гигабитные, однако схематехнические они были соеденены только двумя дифпарами. По этой причине линк через некоторое время падал. Добавил в загрузчик и в драйвер Linux возможность запрета установки связи на гигабит.
\item{Участвовал в изготовлении опытного образца для системы балансировки заряда Li-ion батарей для использования в электробусе. Разрабатывал ПО для контроллера Freescale K60, опытный образец был изготовлен.}
\end{itemize}
}

%------------------------------------------------

\jobdetail
{ООО "НПО АТС", ООО "Компания "АЛСиТЕК"}
{Инженер программист}
{Занимался разработкой ПО для телефонных станций.

Решенные задачи:
\begin{itemize}
\item{Создание SNMP сервера для мониторинга АТС на системе k095. Сервер написан на базе mini\_snmp, Поддерживаеются GET, SET, GETNEXT, TRAP и INFORM запросами.}
\item{Создание k095\_network плагина для k095 системы(Система мониторинга), предназначена для удаленного управления сетевыми настройками через конфиги etcnet. }
\item{Реализовал поддержку протокола mskjson в клиенте системы мониторинга k095. Протокол основан на JSON, но имеет ряд отличий.}
\item{Тестирование и доработка ПО мониторинга телефонной станции (k095), ПО было выполнено на C++, QT.}
\item{Разработка ПО для опытного образца устройства компрессии голосовой информации на базе DSP Freescale.}
\end{itemize}
}

%------------------------------------------------

\jobdetail
{ЗАО "НПКПО"}
{Инженер программист}
{Разрабатывал ПО для промышленного оборудования

Решенные задачи:
\begin{itemize}
\item{САПР для написание обрабатывающих программ для 2 осевых фрезерных станков}
\item{Драйвер для Linux , для работы с "железом" станка с ЧПУ серии "ФПР"}
\item{Система оптического распознавания профиля рамы ПВХ}
\item{ПО для PIC контроллеров платы - контроллера двигателей постоянного тока }
\item{ПО для PIC контроллеров интерфейсной платы ЧПУ. }
\item{ПО для AVR контроллеров семейства ATMEGA для поддержки сети, обновления прошивки, и доступа к ОЗУ контроллера по шине IIC (На шину паралельно подключенно до 128 контроллеров)}
\item{ПО для AVR контроллеров семейства ATMEGA для контроллера дискретного ввода/вывода с подключением на шину IIC }
\item{ПО для AVR контроллеров семейства ATMEGA для контроллера привода c подключением к шине IIC (контроллер поддерживает обратную связь, расчет линейной и круговой интерполяции ы реальном времени) }
\item{Создание платы(Схема, разводка, сборка) дискретного ввода вывода. }
\item{Создание платы(Схема, разводка, сборка, отладка) контроллера привода. }
\item{ПО для AVR контроллера для загрузки прошивки в ПЛИС "Циклон" из внутренней памяти контроллера. }
\item{ПО для ЧПУ, выполняющая программу в G-кодах, и управляющая приводами и дискретными выводами по шине I2C}

\end{itemize}
}


